\documentclass{beamer}
\usepackage{booktabs}
\usetheme{Boadilla}
\setbeamersize{
  text margin left=0.5cm,
  text margin right=0.5cm,
}

\title{Consumption and portfolio choice with recursive preferences and uninsurable labour income risk}
\author{David Kraus}
\institute{École Polytechnique}
\date{April 9, 2025}

\begin{document}

\begin{frame}
\titlepage
\end{frame}

\begin{frame}
\frametitle{Outline}
\tableofcontents
\end{frame}

\section{Motivation}

\begin{frame}
\frametitle{Motivation}

\begin{itemize}
    \item Inequality matters for macroeconomics
    \item Incomplete markets can explain why
    \item Decoupling RRA and IES is arguably also important
    \begin{itemize}
      \item A priori no reason for reciprocal relationship as per power utility
      \item Coupled parameters confound respective effects
      \item Improve ability to quantitatively match empirical moments
      \item Long-run risk + EZ preferences shows promise in resolving asset pricing puzzles (Bansal \& Yaron, 2004)
    \end{itemize}
\end{itemize}

\end{frame}

\section{The model}
\subsection{Model overview}

\begin{frame}
\frametitle{Model overview}
\begin{itemize}
    \item Unit continuum of households
    \item Households face idiosyncratic labour income shocks and an exogenous borrowing constraint (uninsurable labour income risk)
    \item Households can invest in a risk-free and a risky asset
    \item Partial equilibrium approach, i.e., prices are given
    \item Time is continuous, $t\in[0,+\infty)$
    \begin{itemize}
        \item FOCs hold with equality on the interior of the state space
        \item Adjoint relationship between HJB generator and KF
    \end{itemize}
\end{itemize}
\end{frame}

\subsection{Household problem}

\begin{frame}
\frametitle{Household problem}
\textbf{Preferences:} The individual lifetime utility of a household is given by,
$$ V_0=\max_{\{c_t, \theta_t\}_{t\geq0}}{\mathbb{E}_0[\int_0^{\infty}f(c_t,V_t)dt]} $$

Where $f(\cdot)$ denotes the normalized aggregator,
$$ f(c,V) = \frac{\beta}{1 - \sigma} (1 - \gamma) V \left[ \left( \frac{c}{((1 - \gamma) V)^{\frac{1}{1 - \gamma}}} \right)^{1 - \sigma} - 1 \right]$$

Furthermore,  $V_t = \max_{\{c_t, \theta_t\}_{t\geq0}}{\mathbb{E}_0[\int_t^{\infty}f(c_s,V_s)ds]}$

\hfill

\textbf{Idiosyncratic labour income risk:} Labour income follows a two-state Poisson process, where $\tilde{z} \in \{z^L, z^H\}$ with transition intensities $\{\lambda^L, \lambda^H\}$

\end{frame}

\begin{frame}

\textbf{Asset returns:} Safe asset returns are deterministic and follow,
$$dP_t=rP_tdt$$ 
Risky asset returns are stochastic and follow a diffusion process,
$$dQ_t=\mu Q_tdt + \nu Q_tdW_t$$

\textbf{Budget, borrowing and short-sale constraint:} 
$$da_t = (z_t+r_ta_t+\theta_t(\mu-r)a_t-c_t)dt + \theta_ta_t\nu dW_t$$
$$a_t \geq -\phi$$
$$\frac{a_t + \phi}{a_t} \geq \theta_t \geq 0$$

\end{frame}

\begin{frame}
\textbf{Definition:} The household problem in sequence form becomes,
$$\max_{\{c_t, \theta_t\}_{t\geq0}}{\mathbb{E}_0[\int_0^{\infty}f(c_t,V_t)dt]} \text{, such that}$$
$$\tilde{z} \in \{z^L, z^H\} \text{ Poisson with transition intensities } \{\lambda^L, \lambda^H\}$$
$$da_t = (z_t+r_ta_t+\theta_t(\mu-r)a_t-c_t)dt + \theta_ta_t\nu dW_t$$
$$a_t \geq -\phi$$
$$\frac{a_t + \phi}{a_t} \geq \theta_t \geq 0$$

Given $(z_0, a_0) \in \{z^L, z^H\}\times [-\phi,+\infty)$ initial income and wealth

\hfill

\textbf{Solution:} A solution to the household problem is a stochastic process $\{c_t, \theta_t\}_{t\geq0}$

\end{frame}

\subsection{Recursive representation}

\begin{frame}
\frametitle{Recursive representation}

\begin{itemize}
    \item Intuition from dynamic programming: a recursive representation significantly simplifies the problem
    \item In continuous time, solving the problem in sequence form is equivalent to solving a partial differential equation, namely,
    \begin{align*}
    0 = \max_{c, \theta}\{{f(c, V(a,z^j))+\mathbb{E}\frac{dV}{dt}\}}
    \end{align*}
    \item Applying Ito's Lemma this yields,
    \begin{align*}
    0 = \max_{c, \theta}\{f(c, V(a,z^j))+V_a(a,z^j)(z^jw+ra+\theta(\mu-r)a-c) \\ +\frac{1}{2}V_{aa}(a,z^j)a^2\theta^2\nu^2 + \lambda^j(V(a,z^{-j})(a)-V(a,z^j))\}
    \end{align*}
    \item The max operator can be resolved by taking FOCs w.r.t. $c$ and $\theta$ and substituting
\end{itemize}

\end{frame}

\subsection{Boundary conditions}

\begin{frame}
\frametitle{Boundary conditions}
\begin{itemize}
    \item The first boundary condition is provided by (or rather enforces) the borrowing constraint, namely,
    \begin{align*}
    V_a(-\phi, z^j) \geq f_c(\underline{c}^*,  V(-\phi, z^j))
    \end{align*}
    \item The second boundary condition requires some more work. In particular, we use the following proposition,
\end{itemize}

\hfill

\centering
\scalebox{0.8}{
\begin{minipage}{0.8\linewidth}
\textbf{Proposition 1:} \textit{With recursive utility, asymptotic individual policy functions as $a \to \infty$ are given by,}
\begin{align*}
    \theta(a,z^j) &\sim \frac{\mu - r}{\gamma\nu^2} \\
    c(a,z^j) &\sim \left( \frac{\beta}{\sigma} - \frac{(1-\sigma)(\mu-r)^2}{2\gamma\sigma\nu^2} - \frac{r(1-\sigma)}{\sigma} \right)a
\end{align*}
\end{minipage}
}

\hfill

\begin{itemize}
    \item By the FOC w.r.t. $\theta$ this allows us to impose,
    \begin{align*}
         V_{aa}(a,z^j) \sim -\gamma\frac{V_a(a,z^j)}{a} \text{, as } a \to \infty
    \end{align*}
\end{itemize}
\end{frame}

\subsection{The stationary wealth distribution}

\begin{frame}
\frametitle{The stationary wealth distribution}
\begin{itemize}
    \item By the adjoint relationship between the HJB generator and the KF,
    \begin{align*}
        \frac{dg(a,z^j)}{dt}=-\frac{d}{da} [ (z^jw+ra+\theta(a,z^j)(\mu-r)a-c(a,z^j))g(a, z^j) ] \\ + \frac{1}{2}\frac{d^2}{da^2}[a^2\theta(a,z^j)^2\nu^2g(a,z^j)]  -\lambda^jg(z^j,a)+\lambda^{-j}g(z^{-j},a)
    \end{align*}
    \item Finding the stationary wealth distribution simply means solving,
    \begin{align*}
        \frac{dg(a,z^j)}{dt} = 0
    \end{align*}
\end{itemize}
\end{frame}

\section{Numerical method}

\begin{frame}
\frametitle{Numerical method}
\begin{itemize}
    \item Discretize the state space into a matrix of size $I \times 2$
    \item Approximate the first derivative with an upwind scheme and the second derivative with a central difference
    \item Update the value function using a semi-implicit one-step scheme
    \item Consistency + stability + monotonicity $\implies$ convergence to unique viscosity solution (Barles \& Souganidis, 1991)
    \item Solve KF equation "for free" by adjoint relationship, i.e., \\ if $\textbf{A}$ discretizes $\mathcal{A}V=\mathbb{E}\frac{dV}{dt}$ then $\mathbf{A}^\intercal$ discretizes $\mathcal{A}^*g$
    \item Obtain stationary wealth distribution by solving the eigenvalue problem,
    \begin{align*}
        \mathbf{A}^\intercal\mathbf{g} = 0
    \end{align*}
\end{itemize}
\end{frame}

\section{Calibration}

\begin{frame}
\frametitle{Calibration}
\centering
\renewcommand{\arraystretch}{1.2}
\scalebox{0.85}{
\begin{tabular}{l c}

\toprule
Parameter & Value \\
\midrule

\underline{Preferences} \\
$\beta$ (discount rate) & 0.05 \\
$\gamma$ (coefficient of relative risk aversion) & 4 \\
$\psi$ (intertemporal elasticity of substitution) & 0.5 \\

\underline{Labour income} \\
$\{z^L, z^H\}$ (income levels) & $\{0.03, 0.09 \}$ \\
$\{\lambda^L, \lambda^H\}$ (transition intensities) & $\{0.5, 0.5 \}$ \\

\underline{Asset returns} \\
$r$ (risk-free rate) & 0.03 \\
$\mu$ (expected equity returns) & 0.07 \\
$\nu$ (volatility of equity returns) & 0.2 \\
$\phi$ (borrowing constraint) & 0 \\

\end{tabular}
}
\end{frame}

\section{Results}

\begin{frame}
\frametitle{Results}
\begin{figure}[H]
    \centering
    \begin{minipage}{0.45\textwidth}
        \includegraphics[width=\linewidth]{fig5.jpg}
    \end{minipage}
    \hfill
    \begin{minipage}{0.45\textwidth}
        \includegraphics[width=\linewidth]{fig6.jpg}
    \end{minipage}
    \\[1em] % Vertical spacing between rows
    \begin{minipage}{0.45\textwidth}
        \includegraphics[width=\linewidth]{fig7.jpg}
    \end{minipage}
    \hfill
    \begin{minipage}{0.45\textwidth}
        \includegraphics[width=\linewidth]{fig8.jpg}
    \end{minipage} 
    \label{fig:overall}
    \caption{Individual policy functions and the stationary wealth distribution for an increase in risk aversion from $\gamma=4$ to $\gamma=5$}
    \end{figure}
\end{frame}

\begin{frame}
\begin{figure}[H]
    \centering
    \begin{minipage}{0.45\textwidth}
        \includegraphics[width=\linewidth]{fig9.jpg}
    \end{minipage}
    \hfill
    \begin{minipage}{0.45\textwidth}
        \includegraphics[width=\linewidth]{fig10.jpg}
    \end{minipage}
    \\[1em] % Vertical spacing between rows
    \begin{minipage}{0.45\textwidth}
        \includegraphics[width=\linewidth]{fig11.jpg}
    \end{minipage}
    \hfill
    \begin{minipage}{0.45\textwidth}
        \includegraphics[width=\linewidth]{fig12.jpg}
    \end{minipage} 
    \label{fig:overall}
    \caption{Individual policy functions and the stationary wealth distribution for an increase in the intertemporal elasticity of substitution from $\psi=0.4$ to $\psi=0.6$}
    \end{figure}
\end{frame}

\section{Extensions}

\begin{frame}
\frametitle{Extensions}
\begin{itemize}
    \item Cointegrated labour and equity markets to explain increasing risky asset share (Benzoni et al., 2007)
    \item Stock market entry cost to explain limited participation (Gomes \& Michaelides, 2007)
    \item General equilibrium with aggregate risk. This complicates the problem (a lot) and leads to a "monster equation". Moving away \\ from rational expectations could resolve this issue (Moll, 2024)
\end{itemize}
\end{frame}

\begin{frame}[c]
  \centering
  {\usebeamercolor[fg]{title} \usebeamerfont{title} Thank you!}
\end{frame}

\begin{frame}{Some more results}
    \begin{figure}[H]
    \centering
    \begin{minipage}{0.45\textwidth}
        \includegraphics[width=\linewidth]{fig13.jpg}
    \end{minipage}
    \hfill
    \begin{minipage}{0.45\textwidth}
        \includegraphics[width=\linewidth]{fig14.jpg}
    \end{minipage}
    \\[1em] % Vertical spacing between rows
    \begin{minipage}{0.45\textwidth}
        \includegraphics[width=\linewidth]{fig15.jpg}
    \end{minipage}
    \hfill
    \begin{minipage}{0.45\textwidth}
        \includegraphics[width=\linewidth]{fig16.jpg}
    \end{minipage} 
    \label{fig:overall}
    \caption{Individual policy functions and the stationary wealth distribution for an increase in labour income risk}
    \end{figure}
\end{frame}

\begin{frame}
    \begin{figure}[H]
    \centering
    \begin{minipage}{0.45\textwidth}
        \includegraphics[width=\linewidth]{fig17.jpg}
    \end{minipage}
    \hfill
    \begin{minipage}{0.45\textwidth}
        \includegraphics[width=\linewidth]{fig18.jpg}
    \end{minipage}
    \\[1em] % Vertical spacing between rows
    \begin{minipage}{0.45\textwidth}
        \includegraphics[width=\linewidth]{fig19.jpg}
    \end{minipage}
    \hfill
    \begin{minipage}{0.45\textwidth}
        \includegraphics[width=\linewidth]{fig20.jpg}
    \end{minipage} 
    \label{fig:overall}
    \caption{Individual policy functions and the stationary wealth distribution for a time-varying equity premium}
    \end{figure}
\end{frame}

\end{document}
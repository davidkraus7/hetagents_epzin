\documentclass{article}
\usepackage{graphicx}
\usepackage{titlesec}
\usepackage[margin=1in]{geometry}
\usepackage{amsmath}
\usepackage{url}
\usepackage{graphicx}
\usepackage{float}
\usepackage{booktabs}
\usepackage{amssymb}
\graphicspath{ {./images/} }
\titleformat{\section}{\large\bfseries}{\thesection}{1em}{}
\linespread{2}
\setlength{\parindent}{0pt}

\title{Two-asset model with recursive preferences}
\author{David Kraus}
\date{} 

\begin{document}

\maketitle

\section{Income and Returns}

Income $\tilde{z}$ follows a Poisson process that takes values in $\tilde{z} \in \{z^L, z^H\}$ with transition intensities $\{\lambda^L, \lambda^H\}$.

Asset returns for the safe asset are deterministic and follow
$dP_t=rP_tdt$.

Asset returns for the risky asset are stochastic and follow a diffusion process as per $dQ_t=\mu Q_tdt + \nu Q_tdW_t$.

\section{Household Problem in Sequence Form}

As in the Merton model, we define the parameters $a_t=b_tP_t+\pi_tQ_t$ and $\theta_t = \frac{\pi_tQ_t}{a_t}$, such that $\theta_t$ denotes the share of risky assets. The household problem in sequence form then becomes,
$$\max_{c_t, \theta_t}{\mathbb{E}_0[\int_0^{\infty}f(c_t,V_t)dt]}$$ such that, $$da_t = (z_tw_t+r_ta_t+\theta_t(\mu-r_t)a_t-c_t)dt + \theta_ta_t\nu dW_t$$ $$a_t\geq\underline{a}$$

Here, $f(\cdot)$ denotes the normalized aggregator, and $V_t = \max_{c_t, \theta_t}{\mathbb{E}_0[\int_t^{\infty}f(c_s,V_s)ds]}$. In particular,
$$f(c,V) = \frac{\beta}{1 - \sigma} (1 - \gamma) V \left[ \left( \frac{c}{((1 - \gamma) V)^{\frac{1}{1 - \gamma}}} \right)^{1 - \sigma} - 1 \right]$$

\section{Household Problem in Recursive Form}

For now, let us assume that $r_t = r$ and $w_t = w$ in order to obtain a stationary HJB. Using Ito's Lemma and the fact that the flow payoff is not discounted this yields,
$$0=\max_{c, \theta}\{{f(c, V_j(a))+V_j'(a)(z^jw+ra+\theta(\mu-r)a-c)+\frac{1}{2}V_j''(a)a^2\theta^2\nu^2}+\lambda^j(V_{-j}(a)+V_j(a))\}$$

We can resolve the $\max$ operator and obtain the implicit policy functions $c(a)$ and $\theta(a)$ by taking F.O.C.'s with respect to $c$ and $\theta$. Using the functional form of the normalized aggregator $f(\cdot)$, where $1/\sigma$ denotes the IES and $\gamma$ denotes the coefficient of RRA, this yields,
$$c^{-\sigma}\beta ((1-\gamma)V_j(a))^{\frac{\sigma-\gamma}{1-\gamma}}=V_j'(a) \implies c(a) = \left[ \frac{1}{\beta}V_j'(a)((1-\gamma)V_j(a))^{\frac{\gamma-\sigma}{1-\gamma}} \right]^{-1/\sigma}$$
$$(\mu-r)aV_j'(a)+\theta a^2\nu^2V_j''(a) = 0 \implies \theta(a) = -\frac{(\mu - r)V_j'(a)}{a\nu^2V_j''(a)}$$


\section{Boundary Conditions}
The borrowing constraint allows us to impose the following boundary condition,
$$V_j'(\underline{a}) \geq f'(\underline{c}^*, V_j)$$

Here, $\underline{c}^*$ denotes the consumption level of staying put at the borrowing constraint. Note that we need to know $\theta(\underline{a})$ to impose this boundary condition, since $\theta(a)$ determines the consumption level of staying put. We can again circumvent this issue by imposing a no-borrowing constraint, i.e., $\underline{a}=0$.

\hfill

In order to impose the second boundary condition, we use the following proposition. The proposition and its proof are directly analogous to the second part of Proposition 2 in Achdou et al. (2021).

\hfill

\textbf{Proposition 1:} \textit{With recursive utility, asymptotic individual policy functions as $a \to \infty$ are given by,}
$$\theta_j(a) \sim \frac{\mu - r}{\gamma\nu^2}$$
$$c_j(a) \sim \left( \frac{\beta}{\sigma} - \frac{(1-\sigma)(\mu-r)^2}{2\gamma\sigma\nu^2}  -\frac{r(1-\sigma)}{\sigma} \right)a$$

\hfill

In order to prove Proposition 1 we first derive two auxiliary lemmas. Lemma 1 derives an analytical solution for the household problem without income and without the borrowing constraint. Lemma 2 shows a certain homogeneity property of the value function. We can then show that the solution of the household problem presented in Lemma 1 is equivalent to the solution of the household problem as wealth tends to infinity.

\hfill

\textbf{Lemma 1:} \textit{Let $f(\cdot)$ denote the normalized aggregator. Consider the problem,}
\begin{equation}
    0 = \max_{c, \theta}\{ f(c,V(a)) + V'(a)(ra+\theta(\mu-r)a-c)+\frac{1}{2}V''(a)\theta^2\nu^2a^2 \}
\end{equation}
\textit{The optimal policy functions that solve this problem are given by,}
$$\theta(a) = \frac{\mu -r}{\gamma\nu^2}$$
$$c(a)= \left( \frac{\beta}{\sigma} - \frac{(1-\sigma)(\mu-r)^2}{2\gamma\sigma\nu^2}  -\frac{r(1-\sigma)}{\sigma} \right)a$$

\hfill

\textbf{Proof Lemma 1:} Let us first define,
$$H(m,p) := \max_c\{f(c, m)-cp\}$$ 
$$G(p,q):=\max_\theta\{p\theta(\mu-r)a +\frac{1}{2}q\theta^2\nu^2a^2\}$$

Resolving the max operators gives us,
$$H(m,p) = \frac{\sigma}{1-\sigma}\beta^{1/\sigma}((1-\gamma)m)^{\frac{\sigma-\gamma}{\sigma(1-\gamma)}}p^{\frac{\sigma-1}{\sigma}}-\frac{\beta(1-\gamma)}{1-\sigma}m$$
$$G(p,q)=-\frac{p^2(\mu-r)^2}{2q\nu^2}$$

Now, we can rewrite the problem as,
$$0 = H(V(a), V'(a)) + G(V'(a), V''(a))+raV'(a)$$

Let us guess the solution $V(a) = Aa^{1-\gamma}$. Then, $V'(a) = (1-\gamma)Aa^{-\gamma}$ and $V''(a)=-\gamma(1-\gamma)Aa^{-\gamma-1}$. It is straightforward to verify that this solution satisfies the HJB for some $A \in \mathbb{R}$, where $A$ depends on the parameters of the problem.

\hfill

Furthermore, the F.O.C. with respect to consumption yields,
$$V'(a) = \frac{\partial f}{\partial c}(a) \Leftrightarrow c(a) = \beta^{1/\sigma}(1-\gamma)^{\frac{\sigma-1}{\sigma(1-\gamma)}}A^{\frac{\sigma-1}{\sigma(1-\gamma)}}a$$

Putting everything together, we can now write,
\begin{equation*}
\begin{aligned}
    0 &= H(V(a), V'(a))+G(V'(a), V''(a))+raV'(a) \\
    \Leftrightarrow 0 &= Aa^{1-\gamma}\left( \frac{\sigma}{1-\sigma}\beta^{1/\sigma}(1-\gamma)^{\frac{2\sigma-\sigma\gamma-1}{\sigma(1-\gamma)}}A^{\frac{\sigma-1}{\sigma(1-\gamma)}} - \frac{\beta(1-\gamma)}{1-\sigma} + \frac{(1-\gamma)(\mu-r)^2}{2\gamma\nu^2} + r(1-\gamma) \right) \\ 
    \Leftrightarrow 0 &= \frac{\sigma}{1-\sigma}\beta^{1/\sigma}(1-\gamma)^{\frac{2\sigma-\sigma\gamma-1}{\sigma(1-\gamma)}}A^{\frac{\sigma-1}{\sigma(1-\gamma)}} - \frac{\beta(1-\gamma)}{1-\sigma} + \frac{(1-\gamma)(\mu-r)^2}{2\gamma\nu^2} + r(1-\gamma) \\
    \Leftrightarrow 0 &=\frac{\sigma c(a)}{(1-\sigma)a}-\frac{\beta}{1-\sigma}+\frac{(\mu-r)^2}{2\gamma\nu^2}+r
\end{aligned}
\end{equation*}

\hfill

From which we can derive,
$$c(a)=\left( \frac{\beta}{\sigma} - \frac{(1-\sigma)(\mu-r)^2}{2\gamma\sigma\nu^2}  -\frac{r(1-\sigma)}{\sigma} \right)a$$

Now, the F.O.C. with respect to the risky asset share yields,
$$V'(a)(\mu-r)a+V''(a)\theta \nu^2a^2=0$$

Which finally gives us,
$$\theta(a) = -\frac{V'(a)(\mu-r)}{V''(a)\nu^2a} = \frac{\mu - r}{\gamma\nu^2}$$
\hfill $\square$

\hfill

\textbf{Lemma 2:} \textit{Consider the problem,}
\begin{equation}
    0=\max_{c, \theta}\{{f(c, V_j(a))+V_j'(a)(z^jw+ra+\theta(\mu-r)a-c)+\frac{1}{2}V_j''(a)a^2\theta^2\nu^2}+\lambda^j(V_{-j}(a)+V_j(a))\}
\end{equation}
\textit{Then, for any $\xi>0$,} 
$$V_j(\xi a) = \xi^{1-\gamma}V_{\xi,j}(a)$$
\textit{where $V_{\xi,j}$ solves the problem,}
\begin{equation}
    0 = \max_{c, \theta}\{{f(c, V_{\xi,j}(a))+V_{\xi,j}'(a)(\frac{z^jw}{\xi}+ra+\theta(\mu-r)a-c)+\frac{1}{2}V_{\xi,j}''(a)a^2\theta^2\nu^2}+\lambda^j(V_{\xi, -j}(a)+V_{\xi,j}(a))\}
\end{equation}

\newpage

\textbf{Proof Lemma 2:} We consider,
$$V_j(\xi a) = \xi^{1-\gamma}V_{\xi,j}(a)$$

Hence,
$$V_j(a) = \xi^{1-\gamma}V_{\xi,j}(a/\xi)$$
$$V_j'(a) = \xi^{-\gamma}V_{\xi,j}'(a/\xi)$$
$$V_j''(a) = \xi^{-\gamma-1}V_{\xi,j}''(a/\xi)$$

We can plug these expressions into equation $(2)$. We use the same definitions of $H$ and $G$ as in the proof of Lemma 1. This then gives us,
\begin{align*}
    0 &= H(\xi^{1-\gamma}V_{\xi,j}(a/\xi), \xi^{-\gamma}V_{\xi,j}'(a/\xi)) + G(\xi^{-\gamma}V_{\xi,j}'(a/\xi), \xi^{-\gamma-1}V_{\xi,j}''(a/\xi)) \\
      &\quad + \xi^{-\gamma}V_{\xi,j}'(a/\xi)(z^jw+ra) + \lambda^j(\xi^{1-\gamma}V_{\xi,-j}(a/\xi) - \xi^{1-\gamma}V_{\xi,j}(a/\xi))
\end{align*}

Some computation yields,
$$H(\xi^{1-\gamma}V_{\xi,j}(a/\xi), \xi^{-\gamma}V_{\xi,j}'(a/\xi))=\xi^{1-\gamma}H(V_{\xi,j}(a/\xi),V_{\xi,j}'(a/\xi))$$

Similarly,
$$G(\xi^{-\gamma}V_{\xi,j}'(a/\xi), \xi^{-\gamma-1}V_{\xi,j}''(a/\xi))=\xi^{1-\gamma}G(V_{\xi,j}'(a/\xi), V_{\xi,j}''(a/\xi))$$

Plugging in these expressions, dividing both sides by $\xi^{1-\gamma}$, and writing by slight abuse of notation $a=a/\xi$, we then finally obtain,
$$0 = \max_{c, \theta}\{{f(c, V_{\xi,j}(a))+V_{\xi,j}'(a)(\frac{z^jw}{\xi}+ra+\theta(\mu-r)a-c)+\frac{1}{2}V_{\xi,j}''(a)a^2\theta^2\nu^2}+\lambda(V_{\xi, -j}(a)+V_{\xi,j}(a))\}$$
\hfill $\square$

\hfill

\textbf{Proof Proposition 1:} We first derive the asymptotic consumption policy function. From the F.O.C. with respect to consumption and our expression for $V_{\xi, j}$ we obtain,
\begin{align*}
    c_j(a) &= \left[ \frac{1}{\beta}V_j'(a)((1-\gamma)V_j(a))^{\frac{\gamma-\sigma}{1-\gamma}} \right]^{-1/\sigma} = \left[ \frac{1}{\beta}\xi^{-\gamma}V_{\xi,j}'(a/\xi)((1-\gamma)\xi^{1-\gamma}V_{\xi,j}(a/\xi))^{\frac{\gamma-\sigma}{1-\gamma}} \right]^{-1/\sigma} = \xi c_{\xi,j}(a/\xi)
\end{align*}

In particular, for $\xi=a$ this gives us,
$$c_j(a)=c_{\xi,j}(1)\cdot a$$

We let $\xi=a \to \infty$ and obtain,
$$\lim_{a \to \infty} \frac{c_j(a)}{a}=\lim_{\xi\to\infty}c_{\xi,j}(1)=\frac{\beta}{\sigma} - \frac{(1-\sigma)(\mu-r)^2}{2\gamma\sigma\nu^2}  -\frac{r(1-\sigma)}{\sigma}$$

We first note that by Lemma 2, $c_{\xi, j}$ is the consumption policy function solving equation $(3)$. We then note that equation $(3)$ converges to equation $(1)$ as $\xi\to\infty$. Hence, by Lemma 1, we can write the second equality.

\hfill

Hence, we finally have that,
$$c_j(a) \sim \left( \frac{\beta}{\sigma} - \frac{(1-\sigma)(\mu-r)^2}{2\gamma\sigma\nu^2}  -\frac{r(1-\sigma)}{\sigma} \right)a$$

\hfill

Similarly, from the F.O.C. with respect to risky asset share and our expression for $V_{\xi,j}$ we obtain,
$$\theta_j(a)=-\frac{(\mu-r)V_j'(a)}{a\nu^2V_j''(a)}=-\frac{(\mu-r)V_{\xi,j}'(a/\xi)}{\frac{a}{\xi}\nu^2V_{\xi,j}''(a/\xi)}=\theta_{\xi,j}(a/\xi)$$

In particular, for $\xi=a$ this gives us,
$$\theta_j(a)=\theta_{\xi,j}(1)$$

We let $\xi=a \to \infty$ and obtain,
$$\lim_{a \to \infty}\theta_j(a)=\lim_{\xi\to\infty}\theta_{\xi,j}(1)=\frac{\mu-r}{\gamma\nu^2}$$

The second equality follows from Lemma 1 and Lemma 2 by the same argument as before. Hence, we have,
$$\theta_j(a) \sim \frac{\mu - r}{\gamma\nu^2}$$

\hfill $\square$

\hfill

We will use the following property to impose our second boundary condition. By Proposition 2 and the F.O.C. with respect to the risky asset share we have that as $a\to\infty$,
$$\theta(a)=-\frac{(\mu-r)V_j'(a)}{a\nu^2V_j''(a)} \sim \frac{\mu - r}{\gamma\nu^2} \Leftrightarrow V''_j(a) \sim -\gamma\frac{V_j'(a)}{a}$$

\hfill

We can solve for the stationary distribution by solving the KF equation with  reflecting barriers at both boundaries of the state space. We can solve for the general equilibrium by imposing a market clearing condition for the asset market.

\end{document}